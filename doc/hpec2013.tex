\documentclass[conference]{IEEEtran}
%\hyphenation{op-tical net-works semi-conduc-tor}

\begin{document}
\title{Novel Algebras for Advanced Analytics in Julia}

\author{\IEEEauthorblockN{Jeremy Kepner\IEEEauthorrefmark{1},
Viral B. Shah\IEEEauthorrefmark{2},
Jeff Bezanson\IEEEauthorrefmark{3}, 
Stefan Karpinski\IEEEauthorrefmark{4} and
Alan Edelman\IEEEauthorrefmark{5}}
\IEEEauthorblockA{\IEEEauthorrefmark{1}Email: kepner@ll.mit.edu}
\IEEEauthorblockA{\IEEEauthorrefmark{2}Email: viral@mayin.org}
\IEEEauthorblockA{\IEEEauthorrefmark{3}Email: jeff.bezanson@gmail.com}
\IEEEauthorblockA{\IEEEauthorrefmark{4}Email: stefan@karpinski.org}
\IEEEauthorblockA{\IEEEauthorrefmark{5}Email: edelman@math.mit.edu}}

% make the title area
\maketitle

\begin{abstract}
%\boldmath
Put Abstract here.
\end{abstract}

\section{Introduction}

\subsection{Semiring algebra}

The duality between the canonical representation of graphs as abstract
collections of vertices and edges and a sparse adjacency matrix
representation has been a part of graph theory since its inception
~\cite{Konig1931}, ~\cite{Konig1936}. Matrix algebra has been
recognized as a useful tool in graph theory for nearly as long
(see~\cite{Harary1969} and references therein).  A linear algebraic
based approach to graph algorithms that exploits the sparse adjacency
matrix representation of graphs can provide a variety of
benefits. These benefits include syntactic simplicity, easier
implementation, and higher performance.  One way to employ linear
algebra techniques for graph algorithms is to use a broader definition
of matrix and vector multiplication. One such broader definition is
that of a semiring.  For example, in semiring notation we write
matrix-matrix multiply as:

$$C = A +.* B$$

where $$+.*$$ denotes standard matrix multiply.  In such notation, a
semiring requires that addition and multiplication are associative,
addition is commutative, and multiplication distributes over addition
from both left and right.  In graph algorithms, a fundamental
operation is matrix-matrix multiply where both matrices are sparse.
This operation represents multi source 1-hop breadth first search
(BFS) and combine, which is the foundation of many graph
algorithms~\cite{shahgilbert}.  In addition, it is often the case that
operations other than standard matrix multiply are desired, for
example:

\begin{enumerate}
\item $$C = A {max.+} B$$
\item $$C = A min.max B$$
\item $$C = A |.\& B$$
\item $$C = A f().g() B$$
\end{enumerate}

With this more general case of sparse matrix multiply, a wide range of graphs algorithms can be implemented~\cite{KepnerGilbertBook}.

\section{Application example}

A classic example of the utility of the semiring approach is in
finding the minimum path between all vertices in graph (see
~\cite{Rader} in ~\cite{kepnergilbertbook}.  Given a weighted
adjacency matrix for a graph where $A(i,j) = w_{ij}$ is the weight of
a directed edge from vertex $i$ to vertex $j$.  Let $C(i,j)_2$ be the
minimum 2-hop cost from vertex $i$ to vertex $j$ (if such a path
exists).  $C_2$ can be computed via the semiring matrix product:

$$C_2 = A min.+ A$$

Likewise, $C_3$ can be computed via

$$C_3 = A min.+ A min.+ A$$

and more generally

$$C_k = A^k$$

\section{Julia}

Despite advances in compiler technology and execution for
high-performance computing, programmers continue to prefer high-level
dynamic languages for algorithm development and data analysis in
applied math, engineering, and the sciences. High-level environments
such as \Matlab\!, Octave~\cite{Octave}, R~\cite{Rlang},
SciPy~\cite{numpy}, and SciLab~\cite{scilab} provide greatly increased
convenience and productivity. However, C and Fortran remain the gold
standard languages for computationally-intensive problems because
high-level dynamic languages still lack sufficient performance. As a
result, the most challenging areas of technical computing have
benefited the least from the increased abstraction and productivity
offered by higher level languages.

Two-tiered architectures have emerged as the standard compromise between
convenience and performance: programmers express high-level logic in a
dynamic language while the heavy lifting is done in C and Fortran. The
aforementioned dynamic technical computing environments are all themselves
instances of this design. While this approach is effective for some
applications, there are drawbacks. It would be preferable to write
compute-intensive code in a more productive language as well. This is
particularly true when developing parallel algorithms, where code
complexity can increase dramatically. Instead, there is pressure to write
``vectorized'' code, which is unnatural for many problems and might
generate large temporary objects which could be avoided with explicit
loops. Programming in two languages is more complex than using either
language by itself due to the need for mediation between different type
domains and memory management schemes. Interfacing between layers may add
significant overhead and makes whole-program optimization difficult.
Two-tiered systems also present a social barrier, preventing most users
from understanding or contributing to their internals.

Julia is designed from the ground up to take advantage of modern
techniques for executing dynamic languages efficiently. As a result, Julia
has the performance of a statically compiled language while providing
interactive dynamic behavior and productivity like Python, LISP or Ruby.
The key ingredients of performance are:
\begin{itemize}
\item Rich type information, provided naturally by multiple dispatch;
\item Aggressive code specialization against run-time types;
\item JIT compilation using the LLVM compiler framework~\cite{LLVM}.
\end{itemize}
Although a sophisticated type system is made available to the programmer,
it remains unobtrusive in the sense that one is never required to specify
types. Type information flows naturally from having actual values (and
hence types) at the time of code generation, and from the language's core
paradigm: by expressing the behavior of functions using multiple dispatch,
the programmer unwittingly provides the compiler with extensive type
information.

\section{Semiring algebra in Julia}

Multiple dispatch allows $*(op1,op2)(A,B)$, while $*(A,B)$ remains
traditional matmul for numeric arrays.

One can create a new type and then all of a sudden $(A * B)$
automatically does the new matmul just by redefining scalar $+$, and
scalar $*$ for that type.

\section{Performance}


\section{Conclusion}


\end{document}
